\documentclass[11pt,a4paper]{article}
\usepackage{geometry}
\geometry{left=2.5cm,right=2.5cm,top=2.5cm,bottom=2.5cm}
\usepackage{hyperref}
\usepackage{enumitem}
\usepackage{setspace}
\usepackage{ctex}

\setstretch{1.2}

\title{搜索系统改造提议(草案)}
\author{Memory-Search Project}
\date{\today}

\begin{document}
\maketitle

\tableofcontents
\newpage

\section{背景与现状}

当前 \texttt{/search} 接口的实现流程如下:

\begin{enumerate}
  \item 在 \texttt{search\_views} 表上使用 PostgreSQL Full-Text Search(FTS)召回:
  \begin{itemize}
    \item 使用 \texttt{sv.fts @@ plainto\_tsquery('simple', :q)} 做匹配
    \item 使用 \texttt{ts\_rank\_cd} 计算相关性分数
    \item 使用 \texttt{ts\_headline} 生成高亮片段
  \end{itemize}

  \item 通过 \texttt{JOIN problem\_items pi ON pi.id = sv.item\_id} 补全题目实体信息:
  \begin{itemize}
    \item 标题:\texttt{COALESCE(NULLIF(pi.problem\_text,''), pi.bm25\_text)}
    \item 标签、图片、元数据过滤
  \end{itemize}

  \item Python 层按 \texttt{item\_id} 聚合多行视图:
  \begin{itemize}
    \item 题目分数 = 所有视图行中最大 rank
    \item 每题保留 top3 evidence
    \item 最终返回 top\_n 个题目
  \end{itemize}
\end{enumerate}

该方案实现简单、易于调试,适合项目早期验证。

\section{当前主要问题}

\subsection{LIMIT 是行级而不是 item 级}

SQL 中的 \texttt{LIMIT :k} 作用在 \texttt{search\_views} 的“行”上,而不是“题目”上。
如果同一题目在多个 \texttt{view\_type} 中都命中,会占用多个名额,导致:

\begin{itemize}
  \item 召回结果的题目多样性下降
  \item 少数强题垄断召回池
\end{itemize}

\subsection{单一路召回容易偏科}

当前系统只依赖一条召回通道(FTS + 原始 query):

\begin{itemize}
  \item 用户描述偏“解析/方法”,但题干窗口更强
  \item query 包含年份、地区、别名等需要归一化
  \item 有些需求更像模糊匹配而非全文检索
\end{itemize}

\subsection{不同窗口分数不可比}

未来一旦引入多窗口、多 query rewrite,不同窗口的
\texttt{ts\_rank\_cd} 分布差异很大,直接加权求和会导致尺度灾难。

\section{改造目标}

\begin{itemize}
  \item 召回更稳定:对不同描述方式都有覆盖。
  \item 结果可解释:evidence 能体现命中来源窗口/字段。
  \item 改动可控:复用现有表结构与接口,渐进升级。
  \item 便于调参:可通过权重/窗口开关做 A/B。
\end{itemize}

\section{方案:多窗口召回 + 融合重排}

\subsection{窗口(Window)的定义}

窗口 = 一条召回通道(query 版本 $\times$ 视图范围 $\times$ 检索方式)。

初始建议窗口:

\begin{itemize}
  \item W1:题干窗口(\texttt{problem\_text, diagram\_desc})
  \item W2:解析窗口(\texttt{solution\_outline, method\_chain})
  \item W3:记忆窗口(\texttt{user\_notes, user\_tags})
  \item W4:短 query 窗口(关键词抽取后的 query)
\end{itemize}

\subsection{窗口内输出:item 级 topM}

每个窗口执行一次 SQL,输出:

\begin{itemize}
  \item \texttt{item\_id}
  \item \texttt{window\_score}(如窗口内最大 rank)
  \item 窗口内 topK evidence
  \item 标题、标签、图片
\end{itemize}

原则:每个窗口返回的是“题目级” topM,而不是“行级” topM。

\subsection{融合策略:RRF(名次融合)}

由于不同窗口分数不可比,第一版推荐使用 RRF:

若某 item 在窗口 $i$ 的名次为 $r_i$,则该窗口贡献为:

\[
score_i = \frac{1}{k + r_i}
\]

最终融合分:

\[
FinalScore = \sum_i weight_i \cdot \frac{1}{k + r_i}
\]

其中 $k$ 常取 60,$weight_i$ 为窗口权重。

优点:

\begin{itemize}
  \item 不依赖原始分数尺度
  \item 多窗口共同命中会自然靠前
  \item 实战稳定、参数少
\end{itemize}

\subsection{第二阶段:归一化 + 加权和}

在积累用户日志后,可升级为:

\begin{itemize}
  \item 每窗口分数做归一化(分位数、z-score)
  \item 再加权求和或使用学习排序
\end{itemize}

\section{数据与接口设计}

\subsection{接口保持兼容}

继续返回:

\begin{itemize}
  \item \texttt{item\_id, title, score, evidence, user\_tags, images}
\end{itemize}

建议 evidence 中新增可选字段:

\begin{itemize}
  \item \texttt{window}:来源窗口
  \item \texttt{raw\_rank}:行级 rank
  \item \texttt{window\_score}:窗口内聚合分
\end{itemize}

\subsection{窗口配置化}

窗口可用 Python 配置:

\begin{itemize}
  \item 窗口名
  \item 允许的 \texttt{view\_type}
  \item query 变体
  \item 权重
\end{itemize}

\section{实施路线}

\subsection{Phase 0:稳定现状}

\begin{itemize}
  \item 保证现有接口可跑通
  \item 补齐建表、索引、刷新 \texttt{search\_views} 的脚本
\end{itemize}

\subsection{Phase 1:多窗口,多 SQL}

\begin{itemize}
  \item 多跑几次 SQL(不同窗口)
  \item Python 中按 item 聚合
  \item 用 RRF 融合
\end{itemize}

\subsection{Phase 2:窗口内聚合下推到 SQL}

\begin{itemize}
  \item SQL 中 \texttt{GROUP BY item\_id}
  \item 输出窗口内 item 级分数
\end{itemize}

\subsection{Phase 3:日志与学习}

\begin{itemize}
  \item 增加点击、停留、收藏等埋点
  \item 调窗口权重或学习排序
\end{itemize}

\section{评估指标}

\subsection{离线指标}

\begin{itemize}
  \item Recall@K
  \item MRR / nDCG
  \item 覆盖率(题干型、记忆型、解析型 query)
\end{itemize}

\subsection{在线指标}

\begin{itemize}
  \item 点击率 CTR@K
  \item 首点位置
  \item query 改写次数
  \item 无结果率
\end{itemize}

\section{风险与注意事项}

\begin{itemize}
  \item 第一版避免对原始 rank 直接加权
  \item 控制窗口数量与每窗 topM
  \item 最终 evidence 需跨窗口筛选最有解释力的
  \item meta\_filters 建议作为 hard filter
\end{itemize}

\section{附:当前代码跑通的最短 Checklist}

\begin{enumerate}
  \item 数据库中存在 \texttt{problem\_items} 与 \texttt{search\_views}
  \item \texttt{search\_views} 含:\texttt{item\_id, view\_type, fts, text}
  \item \texttt{fts} 与查询均使用 \texttt{'simple'} 字典
  \item 对 \texttt{fts} 建 GIN 索引
  \item 在 psql 中手动验证 FTS 查询
  \item 再启动 FastAPI 调用 \texttt{/search}
\end{enumerate}

\end{document}